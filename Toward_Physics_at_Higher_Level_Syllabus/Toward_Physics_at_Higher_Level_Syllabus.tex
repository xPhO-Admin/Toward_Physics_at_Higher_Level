\documentclass{article}
\usepackage[utf8]{inputenc}
\usepackage[vietnamese]{babel}
\usepackage{multicol}
\usepackage{mathtools}
\usepackage{amsmath}
\usepackage{amssymb}
\usepackage{eqnarray}
\usepackage{siunitx}
\usepackage{tikz}
    
\setlength{\unitlength}{1cm}

\usepackage{titlesec}

\titleformat*{\section}{\Large\bfseries}
\titleformat*{\subsection}{\large\bfseries}
\titleformat*{\subsubsection}{\bfseries}

\usepackage[a4paper]{geometry}
\geometry{left = 1cm, right=1cm, top=1cm, bottom=1cm, noheadfoot, nomarginpar}
%\setlength{\topskip}{2cm}
%\setlength{\parindent}{2cm}

\pagestyle{empty}

\begin{document}

\begin{center}
    {\Huge\textbf{Danh mục kiến thức nên chuẩn bị cho \\ kỳ thi vào 10 chuyên lý}}
\end{center}

\begin{multicols}{2}
\section{Giới thiệu}
% \subsection{Mục đích của Danh mục}
\textit{\textbf{Danh mục được cập nhật} \today. } \vspace{3mm}

Danh mục này hệ thống các kiến thức xuất hiện trong các đề thi tuyển sinh vào các trường chuyên môn vật lý. 

%Do đặc thù địa phương của kỳ thi, mỗi trường chuyên khác nhau sẽ có một dạng đề thi khác nhau, vì vậy, nhóm viết danh mục này quyết định chọn đề chuyên Khoa học Tự nhiên - Đại học Quốc Gia Hà Nội làm mẫu được sử dụng trong danh mục. Theo đánh giá của nhóm, đây là một mẫu đề thi tốt. Theo quan điểm chủ quan của nhóm, đề tuyển sinh chuyên Khoa học Tự nhiên - Đại học Quốc Gia Hà Nội là đề thi có thể nói là khó nhất trong hệ thống các trường chuyên của Việt Nam, đề thi này nặng về mô hình và tính vật lý của bài toán, không lạm dụng những kiến thức không học trước thì không thể biết ở các bậc học cao hơn.

Do đặc thù địa phương của kỳ thi, mỗi trường chuyên khác nhau sẽ có một dạng đề thi khác nhau. Danh mục này không thể bao quát được mọi thứ, không đúng với mọi trường chuyên mà sẽ chỉ khoanh vùng tương đối rộng các kiến thức có trong các kỳ thi tuyển sinh từ vào 10 cũng như một số kiến thức, phương pháp mà chúng tôi khuyến khích các bạn nên tìm hiểu để cải thiện trình độ của mình. Song danh mục vẫn sẽ đảm bảo không đưa vào những phần đòi hỏi kiến thức nền quá lớn mà học sinh trung học cơ sở không thể hiểu hết được như cơ vật rắn, nhiệt học các chất khí lý tưởng, lực điện từ,... tránh tình trạng các học sinh áp dụng công thức một cách máy móc mà không hiểu được bản chất vật lý của nó.

Các bài toán trong đề tuyển sinh thường là các bài toán định lượng, song các kiến thức định tính nằm trong chương trình trung học cơ sở vẫn hoàn toàn có thể xuất hiện. Các kiến thức có trong sách giáo khoa sẽ chỉ được nhắc lại trong danh mục mà không phân tích kỹ hơn.

\section{Toán cho vật lý}

\subsection{Đại số sơ cấp}

\noindent \textbf{Hàm số:} Các kiến thức cơ bản về đơn thức, đa thức. Các hàm số cơ bản như tổng, hiệu, tích, thương, lũy thừa, căn thức, lượng giác cơ bản như $\sin$, $\cos$, $\tan$, $\cot$,... (có thể tìm hiểu sơ qua về hàm logarit).

\noindent \textbf{Phương trình:} Giải phương trình, hệ phương trình bậc nhất, giải phương trình bậc 2, định lý Viète, giải phương trình mũ, lũy thừa, logarit (biết về cách sử dụng máy tính để tìm logarit).

\noindent \textbf{Cực trị:} Các kiến thức về hàm đồng biến, nghịch biến. Tìm cực trị của biểu thức thông qua biệt thức Delta, bất đẳng thức AM–GM, bất đẳng thức Cauchy–Bunyakovski–Schwarz, bất đẳng thức tam giác,...

\subsection{Hình học}

\noindent \textbf{Hình học phẳng:} Đường thẳng, tia, đoạn thẳng; các hình cơ bản: hình tam giác, tam giác cân, tam giác đều hình tứ giác lồi, hình thang, hình thoi, hình vuông, đa giác đều, hình tròn, đường tròn,... và các tính chất: tam giác bằng nhau, tam giác đồng dạng, tứ giác nội tiếp đường tròn,..., hệ thức lượng trong tham giác, hệ thức lượng giác trong tam giác vuông, định lý hàm $\sin$ và định lý hàm $\cos$.

\noindent \textbf{Hình học ba chiều:} Thể tích các hình cơ bản: hình lập phương, hình hộp chữ nhật, hình trụ, hình cầu, hình lăng trụ, hình chóp,...

\subsection{Đồ thị hàm số và hệ tọa độ Decartes, đại lượng vector}

\noindent \textbf{Các đồ thị hàm số cơ bản:} Đồ thị đường thẳng, đường Parabol, đường tròn,... thể hiện qua hệ tọa độ Decartes hai chiều (hệ tọa độ $xOy$).

\noindent \textbf{Vector:} Định nghĩa các đại lượng vector; phương, chiều, điểm đặt và độ lớn; phân tích một vector thành 2 thành phần và chiếu vector lên các trục; tổng hợp vector,...

\subsection{Các kiến thức đặc biệt khác}

\noindent \textbf{Phương trình sai phân:} Sử dụng giả thiết quy nạp để chứng minh các chuỗi. Giải phương trình và hệ phương trình sai phân tuyến tính dạng đơn giản.

\noindent \textbf{Các chuỗi tổng thường gặp:}
$$\sum_{i=0}^n i = 0+1+2+...+n = \dfrac{n(n+1)}{2}.$$
$$\sum_{i=0}^n i^2 = 0^2+1^2+2^2+...+n^2 = \dfrac{n(n+1)(2n+1)}{6}.$$
$$\sum_{i=0}^n x^i = 1+x+x^2+...+x^n = \dfrac{1-x^{n+1}}{1-x}.$$

\noindent \textbf{Tính tổng các phần nhỏ của hàm biến đổi tuyến tính bằng diện tích hình tam giác hoặc hình thang, lấy trung bình:}

Khi cần tính tổng $S$ của những đoạn rất nhỏ của một đại lượng có độ biến đổi $y$ theo $x$, phụ thuộc vào biến $x$ theo dạng tuyến tính (ví dụ như tính quãng đường chuyển động của một vật chuyển động nhanh dần đều, tính nhiệt lượng khi biết công suất tăng dần theo thời gian,...), ta chia phần lấy tổng thành từng đoạn $\Delta x$ rất nhỏ rồi tính tổng các $y \Delta x$ ở từng giá trị của $x$ trong đoạn tính tổng. Mỗi đoạn $y \Delta x$ này sẽ là diện tích của một hình chữ nhật rất bé phía dưới đồ thị (như hình vẽ), và khi lấy tổng của diện tích của các hình chữ nhật rất nhỏ này, ta thu được một hình thang vuông, diện tích hình thang này chính là tổng ta cần tìm $S= \dfrac{(y_1+y_2)}{2}(x_2-x_1)$. Với $\overline{y}=\dfrac{y_1+y_2}{2}$ là giá trị trung bình của $y$ trong khoảng lấy giá trị, ta có thể đưa đến một kết quả trực quan hơn đó là $S=\overline{y}(x_2-x_1)$ (giống như trong quá trình chuyển động nhanh dần đều, ta tính quãng đường bằng vận tốc trung bình nhân với thời gian chuyển động). Tổng quát hơn cho cả những trường hợp $y$ không thay đổi tuyến tính theo $x$ chính là phép tích phân trong giải tích, song ta sẽ không đề cập tới nó trong phần kiến thức của cấp trung học cơ sở.

%Khi ta cần tính tổng của một đại lượng $y$ biến đổi tuyến tính theo $x$ theo dạng $y=ax+b$ trong đoạn từ $x_1$ đến $x_2$, ta thường sử dụng giá trị trung bình của $y$ khi $x$ chạy từ $x_1$ đến $x_2$ rồi nhân với khoảng giá trị $x_2-x_1$ và thu được tổng bằng $\sum y_i \Delta x_i = \dfrac{y_1+y_2}{2} \left( x_2 - x_1 \right) = \left[ a \dfrac{x_2+x_1}{2}+b \right] \left( x_2 - x_1 \right)$. Một điều may mắn đó là cách làm này đưa ta đến một đáp án đúng với trường hợp hàm biến đổi tuyến tính. Khi ta lấy tổng của rất nhiều $y_i \Delta x$, tổng này chính là tổng của vô hạn các hình chữ nhật như ở đồ thị dưới, với $\Delta x$ rất nhỏ thì tổng diện tích hình chữ nhật này sẽ tiến đến bằng giá trị diện tích hình thanh vuông phía dưới đồ thị và bằng công thức diện tích hình thang, ta tìm được biểu thức y như cách lấy trung bình. (đọc thêm về tích phân để hiểu ý tưởng tổng quát của cách lấy tổng này).

\begin{center}
\begin{tikzpicture}[>=stealth]
    %Vẽ hệ trục toạ độ 
    \draw[->](0,0)--(6.0,0);
    \draw[->](0,0)--(0,4.0);
    \draw (6.0,0) node[below]{$x$} (0,4.0) node[left]{$y$} (0,0) 
    node[below left]{$O$};
    %Vẽ đồ thị
    \draw[thick][-] (0.6,1.9)--(5.4,3.1);
    \draw[dashed] (1.0,2.0)--(1.0,0);
    \draw[dashed] (5.0,3.0)--(5.0,0);
    \draw (1.0,0) node[below]{$x_1$} (5.0,0) node[below]{$x_2$};
    \draw[dashed] (1.0,2.0)--(0,2.0);
    \draw[dashed] (5.0,3.0)--(0,3.0);
    \draw (0,2.0) node[left]{$y_1$} (0,3.0) node[left]{$y_2$};
    \draw[blue, fill=blue!10, thick] (1.00,0) rectangle (1.20,2.00);
    \draw[blue, fill=blue!10, thick] (1.20,0) rectangle (1.40,2.05);
    \draw[blue, fill=blue!10, thick] (1.40,0) rectangle (1.60,2.10);
    \draw[blue, fill=blue!10, thick] (1.60,0) rectangle (1.80,2.15);
    \draw[blue, fill=blue!10, thick] (1.80,0) rectangle (2.00,2.20);
    \draw[blue, fill=blue!10, thick] (2.00,0) rectangle (2.20,2.25);
    \draw[blue, fill=blue!10, thick] (2.20,0) rectangle (2.40,2.30);
    \draw[blue, fill=blue!10, thick] (2.40,0) rectangle (2.60,2.35);
    \draw[blue, fill=blue!10, thick] (2.60,0) rectangle (2.80,2.40);
    \draw[blue, fill=blue!10, thick] (2.80,0) rectangle (3.00,2.45);
    \draw[blue, fill=blue!10, thick] (3.00,0) rectangle (3.20,2.50);
    \draw[blue, fill=blue!10, thick] (3.20,0) rectangle (3.40,2.55);
    \draw[blue, fill=blue!10, thick] (3.40,0) rectangle (3.60,2.60);
    \draw[blue, fill=blue!10, thick] (3.60,0) rectangle (3.80,2.65);
    \draw[blue, fill=blue!10, thick] (3.80,0) rectangle (4.00,2.70);
    \draw[blue, fill=blue!10, thick] (4.00,0) rectangle (4.20,2.75);
    \draw[blue, fill=blue!10, thick] (4.20,0) rectangle (4.40,2.80);
    \draw[blue, fill=blue!10, thick] (4.40,0) rectangle (4.60,2.85);
    \draw[blue, fill=blue!10, thick] (4.60,0) rectangle (4.80,2.90);
    \draw[blue, fill=blue!10, thick] (4.80,0) rectangle (5.00,2.95);
\end{tikzpicture}
\end{center}

\section{Cơ học}

\subsection{Động học}

Quãng đường, vận tốc, thời gian, chuyển động thẳng đều, chuyển động đều (có thể tìm hiểu qua về chuyển động biến đổi đều). Hệ quy chiếu quán tính và công thức cộng vận tốc trong đổi hệ quy chiếu.

\subsection{Một số lực trong tự nhiên}

Trọng lực $P=mg$ với $g=\SI{9.81}{m/s^2}\approx \SI{10}{m/s^2}$, lực ma sát, lực đàn hồi tỷ lệ với độ biến dạng $F=k \Delta l$ với $k$ là hệ số đàn hồi và $\Delta l$ là độ biến dạng có hướng ngược lại với chiều biến dạng (định luật Hooke). Các lực liên kết như phản lực và lực căng dây. Chỉ cần nắm định tính về hướng của các lực điện từ.

\subsection{Tính quán tính của chuyển động}

Phát biểu "Khi vật chịu tác dụng của các lực cân bằng thì vật đang đứng yên sẽ tiếp tục đứng yên, đang chuyển động sẽ tiếp tục chuyển động thẳng đều".

\subsection{Tĩnh học}

Cân bằng lực. Các kiến thức về moment lực và cân bằng moment lực như cánh tay đòn, cân bằng đòn bẩy. 

Các điều kiện cân bằng cho các trường hợp riêng:
\begin{itemize}
    \item Điều kiện cân bằng cho một vật chịu tác dụng của hại lực là hai lực đó phải cùng giá, cùng độ lớn và ngược chiều.
    \item Điều kiện cân bằng của một vật chịu tác dụng của ba lực không song song là ba lực đó phải đồng phẳng và đồng quy, hợp lực của hai lực phải cân bằng với lực thứ ba.
    \item Điều kiện cân bằng của một vật có trục quay cố định là tổng moment lực xu hướng làm vật quay theo chiều kim đồng hồ bằng tổng các moment lực có xu hướng làm vật quay ngược chiều kim đồng hồ.
    \item Điều kiện cân bằng của một vật có mặt chân để là giá của trọng lực phải xuyên qua mặt chân đế (hay trọng tâm phải "rơi" vào mặt chân đế).
\end{itemize}

\subsection{Công cơ học}

Công cơ học, động năng (một cách định tính) và thế năng trọng trường (có tính đến định lượng). Định luật bảo toàn và chuyển hóa năng lượng. Công suất.

\subsection{Máy cơ đơn giản}

Mặt phẳng nghiêng, đòn bẩy, ròng rọc. Áp dụng định luật bảo toàn cơ năng với các máy cơ đơn giản lý tưởng.

\subsection{Áp suất thủy tĩnh}

Áp suất thủy tĩnh của các chất lưu (chất khí và chất lỏng). Bình thông nhau.

\subsection{Lực đẩy Archimedes}

Lực đẩy Archimedes, các trường hợp vật nổi và chìm dưới tác dụng của lực đẩy Archimedes,...

\section{Nhiệt học}

\subsection{Các khái niệm cơ bản}

Nhiệt độ, nhiệt kế, nhiệt giai và các thang đo, đơn vị đo nhiệt độ. Nhiệt lượng, nhiệt dung, nhiệt dung riêng của các chất. Năng suất tỏa nhiệt của nhiên liệu.

\subsection{Sự giãn nở vì nhiệt của các chất}

Sự giãn nở vì nhiệt của các chất, hệ số giãn nở dài, hệ số giãn nở khối và liên hệ của chúng.

\subsection{Sự truyền nhiệt}

Dẫn nhiệt (định lượng theo một số mô hình truyền nhiệt, công suất truyền nhiệt tỷ lệ thuận với chênh lệch nhiệt độ, diện tích tiếp xúc, tỷ lệ nghịch với độ dài đoạn truyền), đối lưu (định tính), bức xạ nhiệt (định tính).

\subsection{Phương trình cân bằng nhiệt}

Tổng nhiệt lượng mà các vật nhận nhiệt nhận vào bằng tổng nhiệt lượng của các vật tỏa nhiệt nhả ra.

\subsection{Sự chuyển pha}

Khi các chất lưu bắt đầu chuyển pha (tan chảy, đông đặc, bay hơi, ngưng tụ, thăng hoa,...), nhiệt lượng cấp vào không làm thay đổi nhiệt độ mà chỉ tập trung vào cho quá trình chuyển pha của các chất. Các khái niệm nhiệt nóng chảy, nhiệt hóa hơi,... của các chất. Nhiệt độ sôi.

\subsection{Sự bảo toàn năng lượng trong các hiện tượng cơ và nhiệt, động cơ nhiệt}

Các bài toán về chuyển hóa năng lượng.

\section{Dòng điện không đổi}

\subsection{Định luật Ohm}

Sự phụ thuộc của cường độ dòng điện vào hiệu điện thế hai đầu dây dẫn. Khái niệm điện trở. Các bài toán dòng điện một chiều không đổi chỉ gồm 1 nguồn điện

\subsection{Các định luật Kirchhoff}

Định luật Kirchhoff 1 về cường độ dòng điện (nguyên lý bảo toàn điện tích): Tại bất kỳ nút (ngã rẽ) nào trong một mạch điện, thì tổng cường độ dòng điện chạy đến nút phải bằng tổng cường độ dòng điện từ nút chạy đi, hay:
Tổng giá trị đại số của dòng điện tại một nút trong một mạch điện là bằng không.

\noindent Định luật Kirchhoff 2 về điện thế (định luật bảo toàn điện áp trong một vòng, gọn lại là định luật vòng kín): Tổng giá trị điện áp dọc theo một vòng bằng không.

\noindent Các phương pháp dòng nhánh, dòng vòng, thế đỉnh, chồng chập nguồn, chồng chập dòng,...

\subsection{Mạch tương đương}

Các mạch tương đương nối tiếp, song song, mạch tương đương sao - tam giác,... dành cho điện trở. (có thể tìm hiểu về các phương pháp nguồn tương đương như định lý Thévenin và định lý Norton)

\subsection{Nguyên lý đơn điệu Rayleigh}

Điện trở tương đương giữa hai đỉnh của một đoạn mạch không thể giảm khi ta tăng các điện trở thành phần.

\subsection{Sự phụ thuộc của điện trở vào độ dài và tiết diện dây, khái niệm điện trở suất}

Điện trở của một dây dẫn có điện trở suất $\rho$, chiều dài $l$, tiết diện $S$ là $R=\rho\dfrac{l}{S}$.

\subsection{Công suất của dòng điện, định luật Định luật Joule–Lenz về công suất tỏa nhiệt của dòng điện}

Công suất của một đoạn mạch bất kỳ, định luật Định luật Joule–Lenz về công suất tỏa nhiệt của dòng điện. \\
\textit{Ghi chú: Công thức $P=UI$ luôn đúng với mọi đoạn mạch, mọi phần tử của mạch điện; còn định luật Ohm và định luật Joule-Lenz chỉ đúng đối với phần tử điện trở tỏa nhiệt}

\section{Điện từ học}

\subsection{Giới thiệu chung về từ trường}

Nam châm vĩnh cửu, tác dụng của dòng điện - từ trường, từ phổ - đường sức từ, sự nhiễm điện của sắt từ,...

\subsection{Quy tắc nắm tay phải}

Hướng của cảm ứng từ gây ra bởi cuộn dây có dòng điện chạy qua (định tính).

\subsection{Quy tắc bàn tay trái}

Hướng của lực từ tác dụng lên đoạn dây trong từ trường (định tính).

\subsection{Hiện tượng cảm ứng điện từ}

Dòng điện xoay chiều xuất hiện khi có sự biến thiên từ thông (định tính).

\subsection{Động cơ điện}

Cơ chế của động cơ điện một chiều và động cơ điện xoay chiều.

\subsection{Dòng điện xoay chiều}

Dòng điện xoay chiều, máy phát điện xoay chiều, biến áp,...

\section{Quang học}

\subsection{Định luật truyền thẳng ánh sáng}

Phát biểu "Trong một môi trường trong suốt và đồng tính, ánh sáng truyền đi theo đường thẳng".

\subsection{Sự khúc xạ ánh sáng}

Hiện tượng khúc xạ ánh sáng, quan hệ giữa góc tới và góc khúc xạ (định tính).

\subsection{Ảnh thật và ảnh ảo}

Trong quang học, ảnh thật tập hợp các điểm hội tụ của tia sáng đến từ một vật thể. Khác với ảnh ảo, vốn là tập hợp các điểm hội tụ tưởng tượng bằng việc kéo dài các tia sáng phân kỳ, ảnh thật là tập hợp các điểm hội tụ thực sự của các tia sáng, nơi các hạt ánh sáng thực sự đi vào.

\subsection{Gương phẳng}

Gương phẳng cho ta ảnh ảo với vật và có độ lớn bằng vật.
Khoảng cách từ ảnh đến gương bằng khoảng cách từ vật đến gương hay ảnh đối xứng với vật qua gương.

\subsection{Gương cầu lồi và gương cầu lõm}

Các trường hợp, các tia đặc biệt, quang tâm, tiêu điểm vật và tiêu điểm ảnh, tâm gương, tiêu diện, trục phụ và tiêu điểm phụ, độ phóng đại ảnh, các tính chất ảnh của thấu kính hội tụ và thấu kính phân kỳ. Công thức gương cầu $\dfrac{1}{d}+\dfrac{1}{d'}=\dfrac{1}{f}$.

\subsection{Thấu kính hội tụ và thấu kính phân kỳ}

Các trường hợp, các tia đặc biệt, quang tâm, tiêu điểm vật và tiêu điểm ảnh, tiêu diện, trục phụ, tiêu điểm phụ, độ phóng đại ảnh, các tính chất ảnh của thấu kính hội tụ và thấu kính phân kỳ. Công thức gương cầu $\dfrac{1}{d}+\dfrac{1}{d'}=\dfrac{1}{f}$. \\
Một số quang hệ đặc biệt: hệ vô tiêu, thấu kính ghép sát,...

\subsection{Các quang cụ}

Mắt, máy ảnh, kính mắt, kính lúp, kính hiển vi, kính thiên văn,...

\subsection{Ánh sáng trắng và ánh sáng màu}

Ánh sáng trắng và ánh sáng màu, sự phân tích của ánh sáng trắng, sự pha trộn của ánh sáng màu, màu sắc,... (định tính).

\end{multicols}

\vspace{10mm}

\begin{center}
    --------------------------------------------------------- HẾT ---------------------------------------------------------
\end{center}

\end{document}

