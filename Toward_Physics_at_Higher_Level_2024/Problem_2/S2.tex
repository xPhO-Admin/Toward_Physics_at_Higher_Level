\begin{enumerate}
    \item \textbf{Ổn định nhiệt độ tại động cơ:} Để nhiệt độ của động cơ ổn định, chất tải nhiệt ở vòng 1 phải thu nhiệt từ động cơ cũng với công suất $\mathcal{P}$.

    \textbf{Ổn định nhiệt độ của chất tải nhiệt trong vòng 1:} Giả sử nhiệt độ cao nhất của chất tải nhiệt tại vòng 1 là $T_0$. Sau một khoảng thời gian $\Delta t$ nhỏ, có một lượng $L_1 \Delta t$ chất tải nhiệt ở vòng 1 đi tới buồng trao đổi nhiệt ở nhiệt độ $T_0$, $L_1 \Delta t$ ra khỏi buồng trao đổi nhiệt ở nhiệt độ $T$. Như vậy, nhiệt lượng mà chất tải nhiệt trong vòng 1 đã truyền cho chất tải nhiệt trong vòng 2 trong khoảng thời gian $\Delta t$ là:
    \begin{equation}
        \Delta Q= L_1 \Delta t C_1 T_0 - L_1 \Delta t C_1 T.
    \end{equation}
    Do đó công suất truyền nhiệt của vòng 1 cho vòng 2 là:
    \begin{equation}
        \mathcal{P'}=\dfrac{\Delta Q}{\Delta t}=L_1 C_1 (T_0-T).
    \end{equation}
    Để nhiệt độ của chất tải nhiệt trong vòng 1 là ổn định, nhiệt lượng nó thu từ động cơ trong một khoảng thời gian phải bằng nhiệt tỏa ra cho vòng 2 trong một thời gian tương ứng, nên $\mathcal{P'}=\mathcal{P}$. Ta nhận được biểu thức của $T_0$:
    \begin{equation}
        T_0=T+\dfrac{\mathcal{P}}{L_1 C_1}.
    \end{equation}

    \textbf{Ổn định nhiệt độ của chất tải nhiệt trong vòng 2:} Xét một lớp chất tải nhiệt trong vòng 1 tại buồng trao đổi nhiệt, do nhiệt độ của chất giảm tuyến tính dọc theo buồng, nên nhiệt lượng mà lớp chất đó nhận được từ lớp chất đằng sau nó trên một khoảng thời gian bằng với công suất nhiệt mà nó tỏa ra cho lớp chất đằng trước nó trên một khoảng thời gian tương ứng.

    \begin{figure}[!h]
        \centering
        \scalebox{1}{



% Gradient Info
  
\tikzset {_q77e6wcqi/.code = {\pgfsetadditionalshadetransform{ \pgftransformshift{\pgfpoint{0 bp } { 0 bp }  }  \pgftransformrotate{-90 }  \pgftransformscale{2 }  }}}
\pgfdeclarehorizontalshading{_fmzdny4at}{150bp}{rgb(0bp)=(1,0.2,0);
rgb(37.5bp)=(1,0.2,0);
rgb(62.5bp)=(0.31,0.9,0.9);
rgb(100bp)=(0.31,0.9,0.9)}
\tikzset{_hcup74vjq/.code = {\pgfsetadditionalshadetransform{\pgftransformshift{\pgfpoint{0 bp } { 0 bp }  }  \pgftransformrotate{-90 }  \pgftransformscale{2 } }}}
\pgfdeclarehorizontalshading{_6boflkrhi} {150bp} {color(0bp)=(transparent!25);
color(37.5bp)=(transparent!25);
color(62.5bp)=(transparent!25);
color(100bp)=(transparent!25) } 
\pgfdeclarefading{_hpd8pgkav}{\tikz \fill[shading=_6boflkrhi,_hcup74vjq] (0,0) rectangle (50bp,50bp); } 
\tikzset{every picture/.style={line width=0.75pt}} %set default line width to 0.75pt        

\begin{tikzpicture}[x=0.75pt,y=0.75pt,yscale=-1,xscale=1]
%uncomment if require: \path (0,300); %set diagram left start at 0, and has height of 300

%Shape: Rectangle [id:dp2185885188174972] 
\path  [shading=_fmzdny4at,_q77e6wcqi,path fading= _hpd8pgkav ,fading transform={xshift=2}] (270,71.6) -- (340,71.6) -- (340,219.6) -- (270,219.6) -- cycle ; % for fading 
 \draw  [color={rgb, 255:red, 0; green, 0; blue, 0 }  ,draw opacity=0 ] (270,71.6) -- (340,71.6) -- (340,219.6) -- (270,219.6) -- cycle ; % for border 

%Straight Lines [id:da0212006301849037] 
\draw [color={rgb, 255:red, 0; green, 0; blue, 0 }  ,draw opacity=1 ]    (270,70.6) -- (270,220.6) ;
%Straight Lines [id:da2341810130015769] 
\draw [color={rgb, 255:red, 0; green, 0; blue, 0 }  ,draw opacity=1 ]    (340,71.6) -- (340,220.6) ;
%Straight Lines [id:da12568741183524867] 
\draw [color={rgb, 255:red, 142; green, 23; blue, 40 }  ,draw opacity=1 ]   (360,68.6) -- (360,222.6) ;
\draw [shift={(360,224.6)}, rotate = 270] [color={rgb, 255:red, 142; green, 23; blue, 40 }  ,draw opacity=1 ][line width=0.75]    (10.93,-3.29) .. controls (6.95,-1.4) and (3.31,-0.3) .. (0,0) .. controls (3.31,0.3) and (6.95,1.4) .. (10.93,3.29)   ;
%Straight Lines [id:da8980559358913409] 
\draw [color={rgb, 255:red, 1; green, 0; blue, 0 }  ,draw opacity=1 ]   (306,84.6) -- (306,100.6) -- (306,123.6) ;
\draw [shift={(306,125.6)}, rotate = 270] [color={rgb, 255:red, 1; green, 0; blue, 0 }  ,draw opacity=1 ][line width=0.75]    (10.93,-3.29) .. controls (6.95,-1.4) and (3.31,-0.3) .. (0,0) .. controls (3.31,0.3) and (6.95,1.4) .. (10.93,3.29)   ;
%Straight Lines [id:da9368761284836362] 
\draw   [color={rgb, 255:red, 1; green, 0; blue, 0 }  ,draw opacity=1 ] (305,145.6) -- (305,161.6) -- (305,184.6) ;
\draw [shift={(305,186.6)}, rotate = 270] [color={rgb, 255:red, 1; green, 0; blue, 0 }  ][line width=0.75]    (10.93,-3.29) .. controls (6.95,-1.4) and (3.31,-0.3) .. (0,0) .. controls (3.31,0.3) and (6.95,1.4) .. (10.93,3.29)   ;
%Straight Lines [id:da4947984933205438] 
\draw [color={rgb, 255:red, 1; green, 0; blue, 0 }  ,draw opacity=1 ] [dash pattern={on 4.5pt off 4.5pt}]  (340,113.6) -- (270,113.6) ;
%Straight Lines [id:da4467544607652605] 
\draw [color={rgb, 255:red, 1; green, 0; blue, 0 }  ,draw opacity=1 ] [dash pattern={on 4.5pt off 4.5pt}]  (272,161.6) -- (289,161.6) -- (296,161.6) -- (342,161.6) ;

% Text Node
\draw (365,218) node [anchor=north west][inner sep=0.75pt]   [align=left] {$\displaystyle x$};
% Text Node
\draw (366,136) node [anchor=north west][inner sep=0.75pt]   [align=left] {$\displaystyle \boxed{T( x) =T( 0) -kx}$};
% Text Node
\draw (307,94) node [anchor=north west][inner sep=0.75pt]   [align=left] {$\displaystyle \textcolor[rgb]{0,0,0}{\mathcal{P}_{\text{vào}}}$};
% Text Node
\draw (307,164.6) node [anchor=north west][inner sep=0.75pt]   [align=left] {$\displaystyle \textcolor[rgb]{0,0,0}{\mathcal{P}_{\text{ra}}}$};
% Text Node
\draw (144,135) node [anchor=north west][inner sep=0.75pt]   [align=left] {$\displaystyle \boxed{\mathcal{P}_{\text{vào}} =\mathcal{P}_{\text{ra}} \sim k}$};


\end{tikzpicture}}
        \caption{Mô hình hóa sự truyền nhiệt}
        \label{fig:99}
    \end{figure}

    Như vậy, nguyên nhân làm giảm nhiệt độ của lớp chất đó chỉ là trao đổi nhiệt với chất tải nhiệt trong vòng 2, có nhiệt độ thấp hơn. Vì tiết diện buồng là đều, lưu lượng của dòng chất không đổi nên vận tốc chảy của chất tải nhiệt trong vòng 1 tại buồng trao đổi nhiệt là không đổi. Thế nên để nhiệt độ luôn giảm tuyến tính dọc theo buồng, công suất tỏa nhiệt tại mọi vị trí dọc theo buồng phải như nhau. Mà công suất tỏa nhiệt này lại tỉ lệ với độ chênh lệch nhiệt độ giữa chất tải nhiệt của hai vòng 1 và 2 tại điểm mà chúng tiếp xúc nhau, nên nhiệt độ của chất tải nhiệt trong vòng 2 tại buồng trao đổi nhiệt với vòng 1 cũng có nhiệt độ tuyến tính dọc theo buồng.

    Công suất nhiệt mà chất tải nhiệt trong vòng 2 nhận được từ vòng 1:
    \begin{equation}
        \mathcal{P}=K_1(T_0-T_1)=K_1(T-T_2).
    \end{equation}
    Từ đó ta có được biểu thức của $T_1$
    \begin{equation}
    \boxed{
            T_1=T_0-\dfrac{\mathcal{P}}{K_1}=T+\dfrac{\mathcal{P}}{L_1 C_1}-\dfrac{\mathcal{P}}{K_1},}
    \end{equation}
    và $T_2$:
    \begin{equation}
        \boxed{
        T_2=T-\dfrac{\mathcal{P}}{K_1}.
        }
    \end{equation}
    Để nhiệt độ của chất tải nhiệt trong vòng 2 ổn định, nó cũng phải tỏa nhiệt cho vòng 3 với công suất $\mathcal{P}$. Chứng minh tương tự như bước tìm $T_0$, ta tìm được $\boxed{L_2=L_1}$.

    \textbf{Ổn định nhiệt độ của chất tải nhiệt trong vòng 3:} Do nhiệt độ của chất tải nhiệt trong vòng 2 tại buồng trao đổi nhiệt với vòng 3 là tuyến tính nên chứng minh tương tự như tại buồng trao đổi nhiệt giữa vòng 1-2, ta cũng có nhiệt độ của vòng 3 tuyến tính dọc theo buồng. Công suất nhiệt mà chất tải nhiệt trong vòng 3 nhận được từ vòng 2:
    \begin{equation}
        \mathcal{P}=K_2(T_1-T_3)=K_2(T_2-T_4).
    \end{equation}
    Từ đó ta thu được biểu thức của $T_3$:
    \begin{equation}
    \boxed{
        T_3=T_1-\dfrac{\mathcal{P}}{K_2}=T+\dfrac{\mathcal{P}}{L_1 C_1}-\left(\dfrac{\mathcal{P}}{K_1}+\dfrac{\mathcal{P}}{K_2}\right),\\}
    \end{equation}
    và $T_4$:
        \begin{equation}
        \boxed{T_4=T_2-\dfrac{\mathcal{P}}{K_2}=T-\left(\dfrac{\mathcal{P}}{K_1}+\dfrac{\mathcal{P}}{K_2}\right).}
        \end{equation}
    Chứng minh tương tự như bước tìm $T_0$, ta tìm được biểu thức của $L_3$:
    \begin{equation}
        \boxed{L_3=\dfrac{\mathcal{P}}{C_2(T_3-T_4)}=\dfrac{\mathcal{P}}{C_2(T_0-T)}=\dfrac{C_1}{C_2}L_1.}
    \end{equation}

    \item \textbf{Phổ điểm}
    \begin{center}
    \begin{tabular}{|p{10.8cm}|c|}
    \hline
    \multicolumn{1}{|c|}{Nội dung} & Điểm thành phần\\ 
    \hline
    Tìm được nhiệt độ thấp nhất $T_0$ của chất tải nhiệt ở vòng 1 & 0.50 \\
    \hline
    Chứng minh nhiệt độ của vòng 2 tại buồng trao đổi nhiệt với vòng 1 có nhiệt độ tuyến tính dọc theo buồng& 1.00 \\
    \hline         
    Tính được công suất nhiệt mà chất tải nhiệt trong vòng 2 nhận được từ vòng 1 bằng công thức Newton & 0.50 \\
    \hline
    Biểu diễn $T_1$ và $T_2$ theo $T$, $\mathcal P$, $L_1$, $C_1$, $K_1$ & 0.50 \\
    \hline 
    Chứng minh được $L_2=L_1$ & 0.25 \\
    \hline
    Chứng minh nhiệt độ của vòng 3 tại buồng trao đổi nhiệt với vòng 2 có nhiệt độ tuyến tính dọc theo buồng& 0.25 \\
    \hline
    Tính được công suất nhiệt mà chất tải nhiệt trong vòng 3 nhận được từ vòng 2 bằng công thức Newton & 0.25\\
    \hline
    Biểu diễn $T_3$ và $T_4$ theo $T$, $\mathcal P$, $L_1$, $C_1$ $K_1$, $K_2$ & 0.50  \\
    \hline
    Biểu diễn $L_3$ theo $C_1$, $C_2$, $L_1$ & 0.25  \\
    \hline
    \end{tabular}
    \end{center}
\end{enumerate}
