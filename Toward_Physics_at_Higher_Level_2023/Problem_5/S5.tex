Phụt lớp nước lên bề mặt quả bóng. Chỗ nào thủng lỗ sẽ sinh ra bong bóng. 

\ \ 

\textit{Với quan sát thường thức (hoặc từ kiến thức Vật Lý THPT), có thể bạn cũng nhận ra rằng nếu thêm chút xà phòng pha loãng thì bong bóng xuất hiện sẽ càng có kính thước lớn, dễ phát hiện hơn. Bài tập này được lấy cảm hứng từ một bài kiểm tra kỹ thuật thông dụng, và ở trường hợp chúng ta không muốn làm bẩn hệ hay có nước tung tóe vung vãi khắp nơi (như trong phòng thí nghiệm, khi hệ cần kiểm tra đã được đặt trên dụng cụ quan sát), lựa chọn chất lưu tốt hơn sẽ là cồn tinh khiết (ethanol hoặc 2-propanol), do nó không những vẫn có thể tạo ra bong bóng nhỏ kêu lép bép tại vị trí rò, mà còn bốc hơi biến mất rất nhanh.}